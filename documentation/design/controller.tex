\section{Controller}
The controller is responsible for stabilizing the rocket state, attenuating any disturbances, and driving it to some desired value.
In their configuration the canards can only provide a torque around the roll ($x$) axis of the rocket.
This imposes some constraints on the controller design, as the whole state \ref{eq:model-state-vector} is under-actuated.
The model used for controller design is thus simplified to the state \ref{eq:controller-model-state}, using a linear parameter-varying system.

The controller chosen for the canards is linear state feedback, where the feedback gains $K$ are computed by the \textit{Linear Quadratic Regulator} (LQR) method \cite{werner2021, werner2021b}.
The current state is determined by the state estimation algorithm (the Estimator is treated in Section \ref{sec:estimator}), as seen in Figure \ref{fig:controller-loop}.
The loop is a two-degree-of-freedom design, where the reference signal is prescaled by a feedforward (reference) gain $K_{pre}$.
To account for the non-linearity of the rocket dynamics the gains are selected from a gain schedule, which is a look-up table of gains for different parameters of the linear model from Equations \ref{eq:controller-model-state}, 
\ref{eq:controller-model-ss}, and \ref{eq:controller-model-matrices}. 

\begin{figure}[ht]
    \centering
    \resizebox{0.8\textwidth}{!}{
    \newcommand\lw{0.75pt}
% style of elements
\tikzstyle{block} = [draw, fill=white!20, rectangle, 
    minimum height=3em, minimum width=4em, line width = \lw]
\tikzstyle{dblock} = [draw, rectangle, 
    minimum height=4.5em, minimum width=16em, line width = \lw]
\tikzstyle{sum} = [draw, fill=white!20, circle, node distance=1cm, line width = \lw, inner sep = 0.075cm]
\tikzstyle{bullet} = [draw, fill=black!100, circle, node distance=1cm, line width = \lw, inner sep = 0.04cm]
\tikzstyle{input} = [coordinate]
\tikzstyle{output} = [coordinate]
\tikzstyle{branch} = [coordinate, bullet]
\tikzstyle{pinstyle} = [pin edge={to-,thin,black}]


\begin{tikzpicture}[auto, node distance=2.1cm,>=latex']
    % We start by placing the blocks
    \node [input, name=input] {};
    \node [block, right of=input, 
    node distance=3cm] (prescale) {$K_{pre}$};
    \node [sum, right of=prescale, xshift = 1cm] (sum) {};
    \node [branch, right of=sum, xshift = 1cm] (b1) {};
    \node [block, right of=b1, 
    		node distance=2.5cm] (plant) {Rocket};
    \node [branch, right of=plant, xshift = 2cm] (b2) {};
    \node [block, below of=plant, 
    		node distance=2.5cm] (observer) {Estimator};
    \node [block, left of=observer, 
    		node distance=3cm] (feedback) {$K$};
    \node [block, below of=feedback, 
    		node distance=2cm] (scheduler) {Schedule};
    \node [output, right of=b2, xshift = -1cm] (o1) {};

    \node [output, below of=plant, yshift = 0.8cm] (u_line1) {};
    \node [output, right of=observerxshift = -1cm, below of=plant, yshift = 0.8cm ] (u_line2) {};


    % Once the nodes are placed, connecting them is easy. 
    \draw [->, line width = \lw] (input) -- node {$r(t)$} (prescale);
    \draw [->, line width = \lw] (prescale.east) |- (sum);
    \draw [-, line width = \lw] (sum) -- node[pos=1] {$u(t)$}  (b1);
    \draw [->, line width = \lw] (b1.east) |- (plant);
    \draw [-, line width = \lw] (plant.east) |- (b2);
    \draw [->, line width = \lw] (b2) -- node[pos=0] {$y(t)$} (o1);
    \draw [->, line width = \lw] (b2) |- (observer.345);
    \draw [->, line width = \lw] (observer.west) -- node {$\hat x_R(t)$}(feedback);
    \draw [-, line width = \lw] (b1) |- (u_line1);
    \draw [-, line width = \lw] (u_line1) -- (u_line2);
    \draw [->, line width = \lw] (u_line2) |- (observer.15);
    \draw [->, line width = \lw] (feedback.west) -| (sum.south);
    \draw [->, line width = \lw] (observer.south) |- node[above,pos=0.7] {$\hat x_{FC}(t)$}(scheduler.east);
    \draw [->, dashed, line width = \lw] (scheduler.north) -- node {}(feedback.south);
    \draw [->, dashed, line width = \lw] (scheduler.west) -| node {}(prescale.south);

\end{tikzpicture}
}
    \caption[Block diagram of the control loop]{Block diagram of the control loop. 2-DOF estimator-based state feedback \cite{werner2021} with gain scheduling.}
    \label{fig:controller-loop}
\end{figure}

\subsection{Controller roll model}
\label{sec:controller_model}
The dynamic pressure and the canard coefficient $C_L$ are the scheduling variables of the controller.
The concocted state $x_R$ for the roll model (Equation \ref{eq:controller-model-state}) and the flight condition (Equation \ref{eq:controller-model-fc}) are
\begin{align}
    x_R &= \begin{bmatrix} \phi & \omega_x & \delta \end{bmatrix}^T 
    & 
    x_{FC} &= \begin{bmatrix} \bar p & C_L \end{bmatrix}
    \nonumber
\end{align}
The state-space model is (Equations \ref{eq:controller-model-ss})
\begin{align}
    \dot x_R &= A \: x_R + B \delta_u 
    &
    y_R &= C x_R
    \nonumber
\end{align}
with the matrices (Equations \ref{eq:controller-model-matrices})
\begin{align}    
    A(x_{FC}) &= \begin{bmatrix}
        0 & 1 & 0 \\
        0 & L_{\omega_x} & L_\delta \\
        0 & 0 & -\alpha
    \end{bmatrix}
    &
    B &= \begin{bmatrix}
        0 \\ 0 \\ \alpha
    \end{bmatrix}
    &
    C &= \begin{bmatrix} I_3 \end{bmatrix}
    \nonumber
\end{align}
where the roll control derivative is $L_\delta (\bar p,  \,  C_L)$, and the roll damping derivative is $L_{\omega_x} (\bar p, \, C_{L \omega_x})$.
The matrix $C$ combines the states of the system to an output signal, and can be set arbitrarily for response analysis.
For the case $C = \begin{bmatrix} I_3 \end{bmatrix}$, the output signal is the entire state vector $y(t) = x(t)$, for the case $C = \begin{bmatrix} 1 & 0 & 0 \end{bmatrix}$, the ouput signal is $y(t) = x_1(t) = \phi$.

The (approximately in continuous time) closed loop is 
\begin{align}
    \dot x_R &= (A+BK) x_R + B K_{pre} r & y_R &= C x_R
    \label{eq:controller-closedloop}
\end{align}

\subsubsection{Controllability}
To determine if pole placement is possible, which is a necessary condition for LQR \cite{werner2021}, the roll model is checked for controllability.
The controllability matrix is 
\begin{align}
    \mathcal{C} = \begin{bmatrix}
        B & AB & A^2 B 
    \end{bmatrix}
\end{align}
which has full rank for all $L_\delta \neq 0$, i.e. there exists a feasible feedback gain for every flight condition, except for zero velocity ($\lVert v \rVert = 0$) or a zero coefficient of lift ($C_L = 0$).

\subsection{Control law and tuning}
To compute the desired canard angle $u = \delta_u$, the the current rocket roll state as determined by the state estimation is used.
The control law is a state feedback, combined with a feedforward part for the reference signal $r$ as
\begin{align}
    u(t) &= K \hat x_R(t) + K_{pre} r(t)
\end{align}
where the gains $K = \begin{bmatrix} K_\phi & K_{\omega_X} & K_\delta \end{bmatrix}$ are determined by performing the LQR method around a linear model
\begin{align}
    K(\hat x_{FC}) &= -\texttt{lqr}(A(\hat x_{FC}),B,Q_{c},R_{c}) 
\end{align}
with Matlab's \texttt{lqr} function. 
For the discrete controller, the function \texttt{lqrd}($A, B, Q, R, dt$) is used to directly account for discrete dynamics with sampling time $dt$, but using a continuous-time system (A,B). 
To note: for fast sampling times (much higher sampling frequency than the bandwidth of the model), the feedback gains of \texttt{lqr} and \texttt{lqrd} become approximately equal.
The method is tuned by adjusting the weighting matrices $Q_c$ and $R_c$.
As shown in section \ref{sec:controller-theory}, the matrix $Q_c$ contains the weights for penalizing the state error, while the matrix $R_c$ weighs the expended control effort.

The scalar reference gain $K_{pre}$ is computed for a steady state gain of 1, as
\begin{align}
    K_{pre} (\hat x_{FC}) &=  \frac{-1}{C \left[ A(\hat x_{FC}) + B K(\hat x_{FC}) \right]^{-1} B}
\end{align}
where $C$ depends on the variable to be tracked. 
With $C = \begin{bmatrix} 1 & 0 & 0 \end{bmatrix}$, the channel $r \to \phi$ is selected, which results the roll angle tracking the reference.
The channel $r \to \omega_x$ could also be selected via $C = \begin{bmatrix} 0 & 1 & 0 \end{bmatrix}$, this would result in tracking of the roll rate.

The gains $K$ and $K_{pre}$ can be precomputed for each combination of parameters $A(\hat x_{FC})$.
As gain scheduling is used to account for slowly-varying parameters, the gains are be computed for each design point $x_s$, and combined to a look-up table (see Section \ref{sec:controller-scheduling}).
This way the implemented controller does not need to compute the gains with the LQR method, but can just select the gains from the gain scheduling table. 

\subsection{Optimal control theory}
\label{sec:controller-theory}
\textit{This section is here for completeness, it is not relevant for implementation.}
The LQR method uses the weighting matrices $Q$ and $R$ to solve the linear quadratic optimization problem \cite{werner2021b} \footnote{I think this came out better than the estimation theory, please be kind Prof. Werner, Prof. Faulwasser}
\begin{equation}
    \min V = \int_0^{\infty} (x^T(t) Q x(t) + u^T(t) R u(t)) dt
\end{equation}
for which the solution is of the form 
\begin{equation}
    V^* = x^T P x
\end{equation}
In infinite time, $P$ satisfies the algebraic Ricatti equation 
\begin{equation}
    PA + A^TP - PBR^{-1}B^TP + Q = 0
\end{equation}
which results in the control law
\begin{align}
    u^*(t) &= K x(t) \\
    K &= -R^{-1}B^TP
\end{align}
The resulting controller for the linear system is optimal in the trade-off of minimizing the state error $x(t)$ and control effort $u(t)$ \cite{werner2021}.
The closed-loop is guaranteed to be stable \cite{werner2021b}, and has margins of $[\infty, 0.5]$ in gain and $60^\circ$ in phase. 
This guarantee is only valid however for linear systems with full state measurements \cite{doyle1978}.
The control law is still useful for controlling nonlinear and observed systems, as the robustness properties can be partially preserved \cite{werner2021b}.

\subsection{Gain scheduling}
\label{sec:controller-scheduling}
As the models from canard angle to roll rate are nonlinear and highly dependent on some slowly-varying parameters (the \textit{flight conditions}), a gain-scheduling scheme will be implemented. 
The design points $x_s$ of the schedule are characterized by these slowly-varying flight conditions $x_{FC} = \begin{bmatrix} \bar p & C_L \end{bmatrix}$, being the aerodynamic pressure $\bar p$, and the lift coefficient of the canards $C_L$. 
The simplified model is evaluated around these design points using the current state estimate $A(\hat x_{FC})$, to form the matrix $A$.
As the LQR method produces guaranteed robustness margins \cite{doyle1978, werner2021, werner2021b}, the  interpolation between two close design points should always produce a stable control loop.
\begin{figure}[ht]
    \centering
    \includegraphics[width=0.95\textwidth]{images-design/controller_scheduling-surfaces.png}
    \caption[Scheduling surfaces along dynamic pressure and canard coefficient]{Scheduling surfaces along dynamic pressure $\bar p$ and canard coefficient of lift $C_L$. Feedback gains $K_\phi, K_p, K_\delta$ and reference gain $K_\textit{pre}$.}
    \label{fig:controller-surfaces}
\end{figure}

A large variety of design points are chosen, both along the nominal flight profile, but also encompassing a larger flight envelope to accommodate off-nominal trajectories.
Design points for the condition $L_\delta = 0$ are omitted from the tuning process, as the system is not controllable there.
The control gains for all design points are computed beforehand and stored in a look-up table.
In flight, the control algorithms interpolate the gains from the look-up table according to the current state estimate. 
The pre-computed feedback gains for the roll model are shown in Figure \ref{fig:controller-surfaces}, plotted over the scheduling variables $\bar p$ and $C_L$.
To note is the discontinuity at $\bar p = 0$ and especially at $C_L = 0$, where no feasible design points exist. 
The feedback gains are interpolated over this split.


\subsection{Response analysis}
To analyze some aspects of the control loop, the closed-loop responses are analyzed.
The closed-loop is technically a sampled-data system, with continuous time (the rocket) and discrete time (estimator and controller) dynamics.
For a high sampling rate (short time periods between control outputs), the discrete-time and the continuous-time closed-loop are approximately equal. 
The continuous-time dynamics are (from Equation \ref{eq:controller-closedloop})
$ \dot x_R = (A+BK) x_R + B K_{pre} r$ and $ y_R = C x_R $, with $C = \begin{bmatrix} 1 & 0 & 0 \end{bmatrix}$ for the $r \to \phi$ channel.

The weighting matrices are chosen as 
\begin{align}
    Q_c &= \begin{bmatrix}
        10 & & \\ & 1 & \\ & & 10  
        \end{bmatrix}
    &
    R_c &= (1 \cdot 10^{-3})\,  \bar p
\end{align}

The closed-loop step response for a collection of parameters (here varying velocities) is shown in Figure \ref{fig:controller-steps}.

\begin{figure}[ht!]
    \centering
    \includegraphics[width=0.8\textwidth]{images-design/controller_steps.png}
    \caption[Closed-loop step responses]{Step responses of the closed-loop system for velocities between $30\mathrm{m/s}$ (green) and $800\mathrm{m/s}$ (red), $l=1000\mathrm{m}$, $C_L = 1.5$. Outputs: 1:$\phi$, 2:$\omega_x$, 3:$\delta$.}
    \label{fig:controller-steps}
\end{figure}

The Bode plots of the closed-loop for a collection of parameters (here varying velocities) is shown in Figure \ref{fig:controller-bodes}.

\begin{figure}[ht!]
    \centering
    \includegraphics[width=0.8\textwidth]{images-design/controller_bodes.png}
    \caption[Closed-loop frequency response]{Frequency response of the closed-loop system for velocities between $30\mathrm{m/s}$ (green) and $800\mathrm{m/s}$ (red), $l=1000\mathrm{m}$, $C_L = 1.5$.
    The response is cut off above the sampling frequency of the processor.}
    \label{fig:controller-bodes}
\end{figure}


